\documentclass[../常见分布.tex]{subfiles}

\begin{document}
\section{Beta分布}

\begin{definition}{The Beta Function}{}
for each positive $\alpha$ and $\beta$, define
\begin{equation}\label{}
B(\alpha, \beta)=\int_{0}^{1} x^{\alpha-1}(1-x)^{\beta-1}dx
\end{equation}
the function $B$ is called the \textit{beta function}.
\end{definition}

\begin{theorem}{}{}
for all $\alpha, \beta > 0$,
\begin{equation}\label{}
B(\alpha, \beta)=\frac{\Gamma(\alpha)\Gamma(\beta)}{\Gamma(\alpha+\beta)}
\end{equation}
\end{theorem}


\begin{definition}{Beta Distributions}{}
Let $\alpha, \beta>0$ and let X be a random variable with p.d.f
\begin{equation}\label{}
f(x|\alpha,\beta)=\begin{cases}
\frac{\Gamma(\alpha+\beta)}{\Gamma(\alpha)\Gamma(\beta)} x^{\alpha-1}(1-x)^{\beta-1}\quad&\text{for $0<x<1$}\\
0\quad &\text{otherwise}
\end{cases}
\right.
\end{equation}
\end{definition}


% 矩
\begin{theorem}{Moments}{}
Suppose that X has the beta distribution with parameters $\alpha$ and $\beta$. Then for each positive integer k,
\begin{equation}\label{}
E(X^k)=\frac{\alpha(\alpha+1)\cdots(\alpha+k-1)}{(\alpha+\beta)(\alpha+\beta+1)\cdots(\alpha+\beta+k-1)}
\end{equation}
特别地,
\begin{equation}\label{}
E(X)=\frac{\alpha}{\alpha+\beta}
\end{equation}
\begin{equation}\label{}
Var(X)=\frac{\alpha\beta}{(\alpha+\beta)^2(\alpha+\beta+1)}
\end{equation}
\end{theorem}
% 矩 证明
\begin{proof}
for $k=1, 2, \dots$
\begin{equation}\label{}
\begin{split}
E(X^k)&=\int_{0}^{1} x^kf(x|\alpha, \beta)dx\\
&=\frac{\Gamma(\alpha+\beta)}{\Gamma(\alpha)\Gamma(\beta)} \int_{0}^{1}x^{\alpha+k-1}(1-x)^{\beta-1}dx\\
&=\frac{\Gamma(\alpha+\beta)}{\Gamma(\alpha)\Gamma(\beta)} \cdot \frac{\Gamma(\alpha+k)\Gamma(\beta)}{\Gamma(\alpha+k+\beta)}
\end{split}
\end{equation}
因此,$E(X)=\frac{\alpha}{\alpha+\beta}$,$E(X^2)=\frac{\alpha(\alpha+1)}{\alpha+\beta+1}$,$Var(X)=E(X^2)-E^2(X)=\frac{\alpha\beta}{(\alpha+\beta)^2(\alpha+\beta+1)}$
\end{proof}

















\end{document}