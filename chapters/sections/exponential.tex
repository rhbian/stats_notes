\documentclass[../常见分布.tex]{subfiles}
\begin{document}
\section{指数分布}
\begin{definition}{Exponential Distributions}{}
Let $\beta>0$. A random variable X has the exponential distribution with parameter $\beta$ if X has a continuous distribution with the $p.d.f.$
\begin{equation}\label{}
f(x|\beta)=\left\lbrace \begin{aligned}
&\beta e^{-\beta x} &\quad\text{for $x > 0$}\\
&0 &\quad\text{for $x \leqslant 0$}
\end{aligned}
\right.
\end{equation}
\end{definition}

\begin{theorem}{}{}
The exponential distribution with parameter $\beta$ is the same as the gamma distribution
with parameters $\alpha = 1$ and $\beta$. If $X$ has the exponential distribution with parameter
$\beta$, then
$E(X)=\frac{1}{\beta} \quad\text{and}\quad Var(X)=\frac{1}{\beta^2}$
and the m.g.f. of $X$ is 
\begin{equation}\label{}
\psi(t)=\frac{\beta}{\beta-t} \quad\text{for $t < \beta$}
\end{equation}
\end{theorem}


\begin{proof}
根据Gamma分布,指数分布是Gamma的一个特例,$Gamma(\alpha=1, \beta)$,因此期望$E(X)=\frac{\alpha}{\beta}=\frac{1}{\beta}$,$Var(X)=\frac{\alpha}{\beta^2}=\frac{1}{\beta^2}$,$\psi(t)=(\frac{\beta}{\beta-t})^\alpha=\frac{\beta}{\beta-t}$
\end{proof}

\begin{theorem}{Memoryless Property of Exponential Distributions}{}
Let $X$ have the exponential distribution with parameter $\beta$, and let $t > 0$. Then for every number $h > 0$,
\begin{equation}\label{}
p(X\geqslant t + h| X \geqslant t) = p(X \geqslant h)
\end{equation}
\end{theorem}

\begin{proof}
for each $t > 0$,
\begin{equation}\label{}
p(X \geqslant t)=\int_{t}^{\infty} \beta e^{-\beta x}dx=e^{-\beta t}
\end{equation}
因此,对于所有的$t>0$以及$h>0$,
\begin{equation}\label{}
\begin{split}
p(X\geqslant t+h| x \geqslant t)&=\frac{p(x \geqslant t+h)}{p(X \geqslant t)}\\
&=\frac{e^{-\beta (t+h)}}{e^{-\beta t}}\\
&=e^{-\beta h}\\
&=p(X\geqslant h)
\end{split}
\end{equation}
\end{proof}


\begin{theorem}{}{}
Suppose that the variables $X_1,\dots, X_n$ form a random sample from the exponential distribution with parameter $\beta$. Then the distribution of $Y_1 = \min\{X_1,\dots , X_n\}$ will be the exponential distribution with parameter $n\beta$.
\end{theorem}

\begin{proof}
for every number $t > 0$,
\begin{equation}\label{}
\begin{split}
p(Y_1 > t)&=p(X_1>t, \dots, X_n>t)\\
&=p(X_1>t)\cdots p(X_n>t)\\
&=e^{-\beta t}\cdots e^{-\beta t}\\
&=e^{-n\beta t}
\end{split}
\end{equation}
\end{proof}










\end{document}