\documentclass[../../main.tex]{subfiles}

\begin{document}
\section{$\chi^2$分布}

\begin{definition}{$\chi^2$分布}{}
For each positive number $m$, the gamma distribution with parameters $\alpha=\frac{m}{2}$, and $\beta=\frac{1}{2}$ is called the $\chi^2$ distribution with $m$ degrees of freedom. 

If a random variable $X$ has the $\chi^2$ distribution with $m$ degrees of freedom, the p.d.f. of $X$ for $x > 0$ is
\begin{equation}\label{}
f(x)=\frac{\beta^\alpha}{\Gamma(\alpha)} x^{\alpha-1}e^{-\beta x}=\frac{1}{2^{m/2}\Gamma(\frac{m}{2})}x^{\frac{m}{2}-1}e^{-\frac{x}{2}}
\end{equation}
Also, $f (x) = 0$ for $x \leqslant 0$.
\end{definition}

\begin{theorem}{Mean and Variance}
If a random variable X has the $\chi^2$ distribution with m degrees of
freedom, then E(X) = m and Var(X) = 2m.
\end{theorem}
m.g.f. of X is
\begin{equation}\label{}
\psi(t)=(\frac{1}{1-2t})^{m/2} \quad\text{for $t<\frac{1}{2}$}
\end{equation}

\begin{theorem}{}{}
If the random variables $X_1, \dots, X_k$ are independent and if $X_i$ has the $\chi^2$ distribution with $m_i$ degrees of freedom $(i = 1, \dots, k)$, then the sum $X_1 + \dots + X_k$ has the $\chi^2$ distribution with $m_1, \dots, m_k$ degrees of freedom.
\end{theorem}
\begin{proof}
证明可用$\Gamma$分布
\end{proof}

\begin{theorem}{}{}
Let $X$ have the standard normal distribution. Then the random variable $Y=X^2$ has the $\chi^2$ distribution with one degree of freedom.
\end{theorem}
\begin{proof}
需要证明$Y\sim \chi(1)$,或者说,服从$Y\sim Gamma(\frac{1}{2}, \frac{1}{2})$

Let $f(y)$ and $F(y)$ denote, respectively, the p.d.f. and the c.d.f. of Y . Also, since X has the standard normal distribution, we shall let $\psi(x)$ and $\Psi(x)$ denote the p.d.f. and the c.d.f. of X. Then for y > 0,
\begin{equation}\label{}
\begin{split}
p(Y\leqslant y)&=p(X^2\leqslant y)\\
&=p(-y^{1/2} \leqslant X \leqslant y^{1/2})\\
&=\Phi(y^{1/2}) - \Phi(-y^{1/2})
\end{split}
\end{equation}

由于$f(y)=F^{'}(y)$以及$\phi(x)=\Phi^{'}(x)$,因此
\begin{equation}\label{}
f(y)=\Phi^{'}(y^{1/2})(\frac{1}{2}y^{-1/2}) - \Phi^{'}(-y^{1/2})(-\frac{1}{2}y^{-1/2})
\end{equation}
\end{proof}
同时,因为$\phi(y^{1/2})=\phi(-y^{1/2})=\frac{1}{\sqrt{2\pi}}e^{-y/2}$,$\Gamma(\frac{1}{2})=\pi^{1/2}$此时,
\begin{equation}\label{}
\begin{split}
f(y)&=\frac{1}{\sqrt{2\pi}}y^{-1/2}e^{-y/2}\\
&\sim Gamma(\frac{1}{2}, \frac{1}{2})\\
&\sim \chi^2(1)
\end{split}
\end{equation}

\begin{corollary}
If the random variables $X_1, \dots, X_m$ are i.i.d. with the standard normal distribution, then the sum of squares $X_1^2+\cdots+X_m^2$ 服从$\chi^2$分布,自由度为$m$。
\end{corollary}
If the random variables X1, . . . , Xm are i.i.d. with the standard normal distribution,2 has the χ 2 distribution with m degrees of
then the sum of squares X12 + . . . + Xm
freedom.



\end{document}