\documentclass[../常见分布.tex]{subfiles}

\begin{document}
\section{正态分布}

\begin{definition}{Normal Distribution}{}
A random variable $X$ has the normal distribution with mean μ and variance σ 2 (−∞ < μ < ∞ and σ > 0) if X has a continuous distribution with the
following $p.d.f.$
\begin{equation}\label{}
f(x|\mu, \sigma^2)=\frac{1}{\sqrt{2\pi}\sigma}e^{-\frac{(x-\mu)^2}{2\sigma^2}}
\end{equation}
for $-\infty < x < \infty$
\end{definition}


\begin{theorem}{}{}
正态分布的概率密度函数积分为1。
\begin{equation}\label{}
\int_{-\infty}^{\infty}f(x|\mu, \sigma^2)dx = 1
\end{equation}
\end{theorem}

\begin{proof}
let $y = \frac{x-\mu}{\sigma}$,then
\begin{equation}\label{}
\int_{-\infty}^{\infty}f(x|\mu, \sigma^2)dx = \int_{-\infty}^{\infty}\frac{1}{\sqrt{2\pi}}e^{-\frac{y^2}{2}}dy
\end{equation}
let $I=\int_{-\infty}^{\infty} e^{-\frac{y^2}{2}}dy$,then
\begin{equation}\label{}
\begin{split}
\begin{WithArrows}
I^2 &= \int_{-\infty}^{\infty}e^{-\frac{y^2}{2}}dy \int_{-\infty}^{\infty}e^{-\frac{z^2}{2}}dz\\
&=\int_{-\infty}^{\infty}\int_{-\infty}^{\infty} e^{-\frac{y^2+z^2}{2}}dydz \Arrow{let $y=r\cos\theta $, $z=r\sin\theta $ \\then $dydz=rdrd\theta$}\\
&=\int_{0}^{2\pi}\int_{0}^{\infty} e^{-\frac{r^2}{2}} r drd\theta\\
&=\int_{0}^{2\pi}1 d\theta\\
&=2\pi
\end{WithArrows}
\end{split}
\end{equation}
故有,$\text{原式}=1$
\end{proof}

\begin{theorem}{Moment Generating Function}{}
The m.g.f. of the distribution with p.d.f. given by
Eq. (5.6.1) is
\begin{equation}\label{}
\psi(t) = e^{\mu t + \frac{1}{2}\sigma^2 t^2}
\end{equation}
for $-\infty < t < \infty$
\end{theorem}
\begin{proof}
\begin{equation}\label{}
\begin{split}
\psi(t)&=E(e^{tX})\\
&=\int_{-\infty}^{\infty} \frac{1}{\sqrt{2\pi}\sigma}e^{-\frac{(x-\mu)^2}{2\sigma^2}} e^{tx}dx\\
&=\int_{-\infty}^{\infty} \frac{1}{\sqrt{2\pi}\sigma}e^{tx-\frac{(x-\mu)^2}{2\sigma^2}}dx
\end{split}
\end{equation}
下面来分析$tx - \frac{(x-\mu)^2}{2\sigma^2}$


\begin{equation}
\begin{split}
\begin{WithArrows}
tx-\frac{(x-\mu)^2}{2\sigma^2}&=-\frac{x^2-2\mu x+\mu^2 - 2t\sigma^2 x}{2\sigma^2}\\
&=-\frac{x^2-(2\mu+2t\sigma^2)x+\mu^2}{2\sigma^2} \Arrow{合并成$(x-\mu)^2$的形式}\\
&=-\frac{[x-(\mu+t\sigma^2)]^2+\mu^2 -(\mu+t\sigma^2)^2}{2\sigma^2} \Arrow{简化}\\
&=\mu t+\frac{1}{2}t^2\sigma^2-\frac{[x-(\mu+t\sigma^2)]^2}{2\sigma^2}
\end{WithArrows}
\end{split}
\end{equation}

因此,原式$\psi(t)$为
\begin{equation}
\begin{split}
\begin{WithArrows}
\psi(t) &= e^{\mu t+\frac{1}{2}t^2\sigma^2}\cdot \int_{-\infty}^{\infty} \frac{1}{\sqrt{2\pi}\sigma}e^{-\frac{[x-(\mu+t\sigma^2)]^2}{2\sigma^2}}\\
&=e^{\mu t+\frac{1}{2}t^2\sigma^2}
\end{WithArrows}
\end{split}
\end{equation}
\end{proof}


\begin{theorem}{Mean and Variance}{}
The mean and variance of the distribution with p.d.f. given by
Eq. (5.6.1) are μ and σ 2 , respectively.
\end{theorem}
\begin{proof}
$\psi(t)$的一阶导数和二阶导数为:
\begin{equation*}\label{}
\begin{split}
\psi^{'}(t) &=(\mu+t\sigma^2)e^{\mu t+\frac{1}{2}t^2\sigma^2}\\
\psi^{''}(t)&=([\mu+t\sigma^2]^2+\sigma^2)e^{\mu t+\frac{1}{2}t^2\sigma^2}
\end{split}
\end{equation*}
在$t=0$处,
\begin{equation*}\label{}
\begin{split}
\psi^{'}(0)=\mu\\
\psi^{''}(0)=\mu^2 + \sigma^2
\end{split}
\end{equation*}

因此,
\begin{equation*}\label{}
\begin{split}
E(X)&=\psi^{'}(0)=\mu\\
Var(X)&=\psi^{''}{0}-[\psi^{'}(0)]^2=\sigma^2
\end{split}
\end{equation*}
\end{proof}


\begin{theorem}{Linear Transformations}{}
If X has the normal distribution with mean μ and variance σ 2 and if Y = aX + b,
where a and b are given constants and a = 0, then Y has the normal distribution with
mean aμ + b and variance a 2 σ 2 .
\end{theorem}
\begin{proof}
已知$X$的$m.g.f$为$\psi(t)=e^{\mu t + \frac{1}{2}\sigma^2t^2}$,令$\psi_Y$记作$Y$的$m.g.f$,则有
\begin{equation}\label{}
\begin{split}
\psi_Y(t)&=E(e^{t(aX+b)})\\
&=e^{tb}E(e^{taX})\\
&=e^{tb}\psi(at)\\
&=e^{tb}e^{a\mu t +\frac{1}{2}a^2\sigma^2t^2}\\
&=e^{(a\mu+b)t+\frac{1}{2}a^2\sigma^2t^2}
\end{split}
\end{equation}
因此,均值为$a\mu+b$,方差为$a^2\sigma^2$
\end{proof}


\begin{definition}{Standard Normal Distribution}{}
The normal distribution with mean 0 and variance 1 is
called the \textbf{standard normal distribution}. The p.d.f. of the standard normal distribution
is usually denoted by the symbol $\phi$, and the c.d.f. is denoted by the symbol $\Phi$. Thus,
\begin{equation}\label{}
\phi(x)=f(x|0,1)=\frac{1}{\sqrt{2\pi}e^{-\frac{x^2}{2}}} \quad\text{for $-\infty < x < \infty$}
\end{equation}
and
\begin{equation}\label{}
\Phi(x)=\int_{-\infty}^{x}\phi(u)du \quad\text{for$-\infty < x < \infty$}
\end{equation}
\end{definition}

\begin{theorem}{Consequences of Symmetry}{}
For all x and all 0 < p < 1
\begin{equation}\label{}
\Phi(-x)=1-\Phi(x) \quad and \quad \Phi^{-1}(p) = -\Phi^{-1}(1-p)
\end{equation}
\end{theorem}

\begin{proof}
由于$\phi(x)$是关于y轴的偶函数。因此,对于所有的$x(-\infty<x<\infty)$,$p(X\leqslant x) = p(X \geqslant x)$,即$\Phi(x)=1-\Phi(-x)$

第二个公式,$x=\Phi^{-1}(p)$,$-x=\Phi^{-1}(1-p)$
\end{proof}

\begin{theorem}{Converting Normal Distributions to Standard}{}
Let X have the normal distribution with
mean μ and variance σ 2 . Let F be the c.d.f. of X. Then Z = (X − μ)/σ has the
standard normal distribution, and, for all x and all 0 < p < 1,
\begin{equation}\label{}
F(x)=\Phi(\frac{x-\mu}{\sigma})
\end{equation}
\begin{equation}\label{}
F^{-1}(p)=\mu+\sigma\Phi^{-1}(p)
\end{equation}
\end{theorem}

\begin{proof}
令$Z = \frac{X-\mu}{\sigma}$,
\begin{equation}\label{}
F(x)=p(X \leqslant x)=p(\frac{X-\mu}{\sigma}\leqslant \frac{x-\mu}{\sigma})=p(Z\leqslant \frac{x-\mu}{\sigma})=\Phi(\frac{x-\mu}{\sigma})
\end{equation}
\end{proof}

\begin{theorem}{Linear Combinations of Normally Distributed Variables}{}
If the random variables $X_1, \dots, X_k$ are independent and if $X_i$ has the normal distribution with mean $\mu_i$ and variance $\sigma_i^2(i=1,\cdots,k)$, then the sum $X_1, \cdots, X_k$ has the normal distribution with mean  $\mu_1, \cdots, \mu_k$ and variance $\sigma_1^2, \cdots, \sigma_k^2$.

\end{theorem}
\begin{proof}
已知,$X_i$的m.g.f为$\psi_i(t)=e^{\mu t+\frac{1}{2}\sigma^2t^2}$,设$X_1+\cdots+X_k$的m.g.f为$\psi(x)$。
由于独立性,可得
\begin{equation}\label{}
\begin{split}
\psi(t)&=\prod\limits_{i=1}^{k}\psi_i(t)\\
&=e^{(\sum\limits_{i=1}^{k} \mu_i)t + \frac{1}{2}(\sum\limits_{i=1}^{k}\sigma_i^2)t^2}
\end{split}
\end{equation}
\end{proof}


\begin{definition}{Sample Mean}{}
Let $X_1, \dots, X_n$ be random variables. The average of these $n$ random
variables, $\frac{1}{n}\sum\limits_{i=1}^{n}X_i$, is called their sample mean and is commonly denoted $\bar{X}_n$.
\end{definition}

\begin{corollary}{}{}
Suppose that the random variables $X_1, \dots , X_n$ form a random sample from the
normal distribution with mean $\mu$ and variance $\sigma^2$ , and let $\bar{X}_n$ denote their sample
mean. Then $\bar{X}_n$ has the normal distribution with mean $\mu$ and variance $\frac{\sigma^2}{n}$.
\end{corollary}



\end{document}






















