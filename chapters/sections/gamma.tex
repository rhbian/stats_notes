\documentclass[../常见分布.tex]{subfiles}

\begin{document}
\section{Gamma分布}

\begin{definition}{The Gamma Function}{}
For each positive number α, let the value $\Gamma(\alpha)$ be defined by
the following integral:
\begin{equation}\label{}
\Gamma(\alpha) = \int\limits_{0}^{\infty}x^{\alpha-1}e^{-x}dx
\end{equation}
The function $\Gamma$ defined by Eq. (5.7.2) for $\alpha > 0$ is called the gamma function.
\end{definition}

\begin{theorem}{}{}
if $\alpha > 1$, then
\begin{equation}\label{}
\Gamma(\alpha)=(\alpha-1)\Gamma(\alpha-1)
\end{equation}
\end{theorem}

\begin{proof}
We shall apply the method of integration by parts to the integral in Eq. (5.7.2).
If we let $u=x^{\alpha-1}$ and $dv=e^{-x}dx$, then $du=(\alpha-1)x^{\alpha-2}dx$ and $v=-e^{-x}$. Therefore,
\begin{equation}\label{}
\begin{split}
\Gamma(\alpha)&=\int_{0}^{\infty}udv=[uv]_0^{\infty}-\int_{0}^{\infty}vdu\\
&=[-x^{\alpha-1}e^{-x}]_0^{\infty} + (\alpha-1)\int_{0}^{\infty}x^{\alpha-2}e^{-x}dx\\
&=0+(\alpha-1)\Gamma(\alpha-1)
\end{split}
\end{equation}
\end{proof}


\begin{theorem}{}{}
For every positive integer n,
\begin{equation}\label{}
\Gamma(n)=(n-1)!
\end{equation}

\begin{proof}
\begin{equation}\label{}
\begin{split}
\Gamma(n)&=(n-1)\Gamma(n-1)\\
&=(n-1)(n-2)\Gamma(n-2)\\
&=(n-1)(n-2)\cdots(1)\Gamma(1)\\
&=(n-1)!
\end{split}
\end{equation}

\end{proof}
\end{theorem}






\end{document}