\documentclass[../main.tex]{subfiles}

\begin{document}
\section{Problems of Testing Hypotheses 假设检验问题}
\begin{definition}{Null and Alternative Hypotheses/Reject.}{}
The hypothesis $H_0$ is called the null hypothesis and the hypothesis $H_1$ is called the alternative hypothesis. When performing a test, if
we decide that $\theta$ lies in $\Omega_1$, we are said to reject $H_0$. If we decide that $\theta$ lies in $\Omega_0$, we are said not to reject $\Omega_0$.
\end{definition}


\begin{definition}{power function}{}
Let $\delta$ be a test procedure. The function $\pi(\theta|\delta)$ is called the power function of the test $\delta$. If $S_1$ denotes the critical region of $\delta$, then the power function $\pi(\theta|\delta)$ is determined by the relation
\begin{equation}\label{}
\pi(\theta|\delta)=p(X\in S_1| \theta) \quad\text{for $\theta \in \Omega$}
\end{equation}
\end{definition}

\begin{definition}{Type I/II Error}{}
An erroneous decision to reject a true null hypothesis is a \textit{type I error},
or an error of the first kind. An erroneous decision not to reject a false null hypothesis is called a \textit{type II error}, or an error of the second kind.
\begin{center}
\begin{tabular}[H]{|c|c|c|}
\hline
Null Hypothesis & True & False \\
\hline
reject & \textbf{I} 类错误 &  \\
\hline
not reject &  &\textbf{II}类错误 \\
\hline
\end{tabular}
\end{center}
\end{definition}

两类错误无法同时避免,我们尽可能的避免第一类错误(这种错误更严重)。即,使拒绝$H_0$出错的概率最低。或者,在一些情况下,有要证明的理论时,把要证明的结果放在备择假设($H_1$)中。


\begin{definition}{Level/Size}{}
A test that satisfies (9.1.6) is called a level $\alpha_0$ test, and we say that the test
has level of significance $\alpha_0$. In addition, the size $\alpha(\delta)$ of a test $\delta$ is defined as follows:
\begin{equation}\label{}
\alpha(\delta)=\sup\limits_{\theta\in\Omega_0} \pi(\theta|\delta)
\end{equation}
\end{definition}

\begin{corollary}{}{}
A test $\delta$ is a level $\alpha_0$ test if and only if its size is at most $\alpha_0$ (i.e., $\alpha(\delta) \leqslant \alpha_0$). If the null hypothesis is simple, that is, $H_0:\theta = \theta_0$, then the size of $\delta$ will be $\alpha(\delta)=\pi(\theta_0|\delta)$.
\end{corollary}











\end{document}