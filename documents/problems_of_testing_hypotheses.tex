\documentclass[../main.tex]{subfiles}

\begin{document}
\section{Problems of Testing Hypotheses 假设检验问题}
假设检验问题专注于讨论一个分布的参数$\theta$是在参数空间的某个子集中,还是在这个子集的补集中。在一维空间中,这个讨论就简化为两个区间。本节中主要专注的是假设检验的方法论。也会给出一个假设检验和置信区间的等式关系。

\begin{definition}{原假设与备择假设}{}
假设$H_0$被成为原假设或零假设(the null hypothesis) ,假设$H_1$被称为备择假设(the alternative hypothesis)。当执行一个假设时,如果我们判定$\theta$在$\Omega_1$中,我们则拒绝原假设$H_0$。如果我们判定$\theta$在$\Omega_0$中,我们则称不拒绝原假设$\Omega_0$。
\end{definition}


\begin{definition}{power function}{}
令$\delta$为一个检验过程。函数$\pi(\theta|\delta)$被称为$\delta$的功效函数(power function)。如果$S_1$记为$\delta$的判别区域(critical region),则the power function $\pi(\theta|\delta)$ 由以下关系来定义
\begin{equation}\label{}
\pi(\theta|\delta)=p(X\in S_1| \theta) \quad\text{for $\theta \in \Omega$}
\end{equation}
如果是在检验统计量$T$和拒绝域$R$中来讨论检验过程$\delta$,则power function为
\begin{equation}\label{}
\pi(\theta|\delta)=p(T\in R|\theta) \quad\text{for $\theta \in \Omega$}
\end{equation}
\end{definition}

\begin{definition}{Type I/II Error}{}
An erroneous decision to reject a true null hypothesis is a \textit{type I error},
or an error of the first kind. An erroneous decision not to reject a false null hypothesis is called a \textit{type II error}, or an error of the second kind.
\begin{center}
\begin{tabular}[H]{|c|c|c|}
\hline
Null Hypothesis & True & False \\
\hline
reject & \textbf{I} 类错误 &  \\
\hline
not reject &  &\textbf{II}类错误 \\
\hline
\end{tabular}
\end{center}
\end{definition}

两类错误无法同时避免,我们尽可能的避免第一类错误(这种错误更严重)。即,使拒绝$H_0$出错的概率最低。或者,在一些情况下,有要证明的理论时,把要证明的结果放在备择假设($H_1$)中。


\begin{definition}{Level/Size}{}
A test that satisfies (9.1.6) is called a level $\alpha_0$ test, and we say that the test
has level of significance $\alpha_0$. In addition, the size $\alpha(\delta)$ of a test $\delta$ is defined as follows:
\begin{equation}\label{}
\alpha(\delta)=\sup\limits_{\theta\in\Omega_0} \pi(\theta|\delta)
\end{equation}
\end{definition}

\begin{corollary}{}{}
A test $\delta$ is a level $\alpha_0$ test if and only if its size is at most $\alpha_0$ (i.e., $\alpha(\delta) \leqslant \alpha_0$). If the null hypothesis is simple, that is, $H_0:\theta = \theta_0$, then the size of $\delta$ will be $\alpha(\delta)=\pi(\theta_0|\delta)$.
\end{corollary}


\begin{definition}{p-value}{}
p值是在所有观测的数据基础上,能够拒绝原假设的最小显著性水平。
\end{definition}

\begin{theorem}{Defining Confidence Sets from Tests}{}
置信区间。令$X=(X_1,\dots,X_n)$为从参数为$\theta$的总体中的随机取样。令$g(\theta)$为函数,假设每一个$g(\theta)$的可能值$g_0$,总存在一个level为$\alpha_0$的$\delta_{g_0}$的原假设。
\begin{equation}\label{}
H_{0, g_0}:g(\theta)=g_0, H_{1, g_0}: g(\theta)\neq g_0
\end{equation}
对于$X$中每一个可能的值$x$,定义:
$\omega(x)={g_0: \delta_{g_0}\text{不拒绝}H_{0, g_0}} \quad\text{if}\quad X=x$被观测到。
\end{theorem}





\end{document}