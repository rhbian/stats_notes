\documentclass[../main.tex]{subfiles}

\begin{document}

\begin{theorem}{Level and Unbiasedness of t Tests}{}
令$X=(X_1, \dots, X_n)$来自于均值为$\mu$,方差为$\sigma^2$的正态分布。令$U$为统计量
\begin{equation}\label{}
U=\dfrac{\sqrt{n}(\bar{X}_n-\mu_0)}{s}
\end{equation}
其中,$s=\sqrt{\dfrac{1}{n-1}\sum\limits_{i=1}^{n}(X_i-\bar{X}_n)^2}$,令$c$为$t(n-1)$分布的$1-\alpha_0$分位数,令$\delta$为检验过程:如果$U\geqslant c$则拒绝$H_0$。则power function $\pi(\mu, \sigma^2|\delta)$有以下性质:

\begin{enumerate}
\item $\pi(\mu, \sigma^2|\delta) = \alpha_0$ when $\mu = \mu_0$ \label{item-1},
\item $\pi(\mu, \sigma^2|\delta) < \alpha_0$ when $\mu < \mu_0$ \label{item-2},
\item $\pi(\mu, \sigma^2|\delta) > \alpha_0$ when $\mu > \mu_0$ \label{item-3},
\item $\pi(\mu, \sigma^2|\delta) \rightarrow 0$ when $\mu \rightarrow -\infty$ \label{item-4},
\item $\pi(\mu, \sigma^2|\delta) \rightarrow 1$ when $\mu \rightarrow \infty \label{item-5}$,
\end{enumerate}
另外,检验过程$\delta$的size为$\alpha_0$并且无偏。
\end{theorem}

\begin{proof}
如果$\mu=\mu_0$,则$U$服从$t$分布,自由度为$n-1$,因此,
\begin{equation}\label{}
\pi(\mu_0, \sigma^2|\delta) = p(U\geqslant c|\mu_0, \sigma^2) = \alpha_0.
\end{equation}
以上,证明了\ref{item-1},对于\ref{item-2}和\ref{item-3},定义:
\begin{equation}\label{}
U^* = \dfrac{n^{1/2}(\bar{X}_n-\mu)}{s} \quad\text{and}\quad W=\dfrac{n^{1/2}(\mu_0-\mu)}{s}
\end{equation}
构建$U=U^*-W$,首先,假设$\mu<\mu_0$,所以$W>0$,以下有:
\begin{equation}\label{}
\begin{split}
\begin{WithArrows}
\pi(\mu, \sigma^2|\delta) &= p(U\geqslant c| \mu, \sigma^2)\\
&= p(U^* -W\geqslant c | \mu, \sigma^2)\\
&= p(U^* \geqslant c+W|\mu, \sigma^2)\Arrow{由于$W>0$}\\
&<p(U^* \geqslant c| \mu, \sigma^2)\Arrow{$U^*\sim t(n-1)$}\\
&=\alpha_0
\end{WithArrows}
\end{split}
\end{equation}
同理,当$\mu > \mu_0$时,即$W<0$,所以可推导出$\pi(\mu, \sigma^2|\delta)>\alpha_0$when$\mu>\mu_0$
\end{proof}

\begin{corollary}{t Tests for Hpotheses}
令$X=(X_1, \dots, X_n)$来自于均值为$\mu$,方差为$\sigma^2$的正态分布。令$U$为统计量
\begin{equation}\label{}
U=\dfrac{\sqrt{n}(\bar{X}_n-\mu_0)}{s}
\end{equation}
其中,$s=\sqrt{\dfrac{1}{n-1}\sum\limits_{i=1}^{n}(X_i-\bar{X}_n)^2}$,令$c$为$t(n-1)$分布的$1-\alpha_0$分位数,令$\delta$为检验过程:如果$U\leqslant c$则拒绝$H_0$。则power function $\pi(\mu, \sigma^2|\delta)$有以下性质:

\begin{enumerate}
\item $\pi(\mu, \sigma^2|\delta) = \alpha_0$ when $\mu = \mu_0$ \label{item-1},
\item $\pi(\mu, \sigma^2|\delta) > \alpha_0$ when $\mu < \mu_0$ \label{item-2},
\item $\pi(\mu, \sigma^2|\delta) < \alpha_0$ when $\mu > \mu_0$ \label{item-3},
\item $\pi(\mu, \sigma^2|\delta) \rightarrow 1$ when $\mu \rightarrow -\infty$ \label{item-4},
\item $\pi(\mu, \sigma^2|\delta) \rightarrow 0$ when $\mu \rightarrow \infty \label{item-5}$,
\end{enumerate}
另外,检验过程$\delta$的size为$\alpha_0$并且无偏。
\end{corollary}

\begin{theorem}{p-value for t test}{}
假设我们在检验一个假设,假设为$\mu\geqslant\mu_0$或$\mu\leqslant\mu_0$,令$U=n^{1/2}\cdot\dfrac{\bar{X}_n-\mu_0}{s}$,令$u$为$U$的观测值。令$T_{n-1}(\cdot)$为$n-1$个自由的$t$分布的$c.d.f.$,假设($H_0: \mu\leqslant\mu_0$)的$p-value$为$1-T_{n-1}(u)$,同时假设($H_0:\mu\geqslant\mu_0$)的$p-value$为$T_{n-1}(u)$。
\end{theorem}
\begin{proof}
令$T_{n-1}^{-1}(\cdot)$表示$n-1$个自由度的$t$分布的分位数。这是$T_{n-1}$是严格的增函数。拒绝原假设$H_0: \mu\leqslant\mu_0$当且仅当$u\geqslant T_{n-1}^{-1}(1-\alpha_0)$,等价于$T_{n-1}(u)\geqslant1-\alpha_0$,等价于$\alpha_0\geqslant1-T_{n-1}(u)$
\end{proof}






\end{document}