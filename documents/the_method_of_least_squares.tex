\documentclass[../main.tex]{subfiles}

\begin{document}
\section{最小二乘法}
\begin{theorem}{最小二乘法}{}
令$(x_1, y_1),\dots, (x_n, y_n)$为$n$个点组成的集合。通过这些点的直线满足最小方差的情况下,斜率(slope)和截距(intercept)分别为:
\begin{equation}\label{Least-Squares}
\begin{split}
\hat{\beta_1}&=\dfrac{\sum\limits_{i=1}^{n}(x_i-\bar{x})(y_i-\bar{y})}{\sum\limits_{i=1}^{n}(x_i-\bar{x})^2}\\
\hat{\beta_0}&=\bar{y}-\hat{\beta_1}\bar{x}
\end{split}
\end{equation}
\end{theorem}

\begin{proof}
根据最小方差原则,目标是使$(y_i-\hat{y_i})^2$最小。
令$Q=[y_i - (\beta_0+\beta_1x_i)]^2$,分别对$\beta_0$和$\beta_1$求偏导,得到
\begin{equation}\label{}
\begin{split}
-2\sum\limits_{i=1}^{n}[y_i-(\beta_0+\beta_1x_i)] &= 0\\
-2\sum\limits_{i=1}^{n}[y_i-(\beta_0+\beta_1x_i)]x_i &=0
\end{split}
\end{equation}
化简可得到
\begin{equation}\label{}
\begin{split}
\sum\limits_{i=1}^{n}y_i &= n\beta_0+\beta_1\sum\limits_{i=1}^{n}x_i \\
\sum\limits_{i=1}^{n}x_iy_i &= \beta_0\sum\limits_{i=1}^{n}x_i+\beta_1\sum\limits_{i=1}^{n}x_i^2
\end{split}
\end{equation}
将$\beta_0=\bar{y}-\beta_1\bar{x}$带入第二个式子中,

\begin{equation}\label{}
\begin{split}
\begin{WithArrows}
\sum\limits_{i=1}^{n}x_iy_i &= (\bar{y}-\beta_1\bar{x})\sum\limits_{i=1}^{n}x_i + \beta_1\sum\limits_{i=1}^{n}x_i^2\\
\sum\limits_{i=1}^{n}x_iy_i &= n\bar{x}\bar{y}-n\beta_1\bar{x}^2+\beta_1\sum\limits_{i=1}^{n}x_i^2\\
\sum\limits_{i=1}^{n}x_iy_i &= n\bar{x}\bar{y}-n\beta_1\bar{x}^2+\beta_1\sum\limits_{i=1}^{n}x_i^2\\
\hat{\beta_1}&=\dfrac{\sum\limits_{i=1}^{n}x_iy_i-n\bar{x}\bar{y}}{\sum\limits_{i=1}^{n}x_i^2-n\bar{x}^2} \Arrow{中心化形式}\\
\hat{\beta_1}&=\dfrac{\sum\limits_{i=1}^{n}(x_i-\bar{x})(y_i-\bar{y})}{\sum\limits_{i=1}^{n}(x_i-\bar{x})^2}
\end{WithArrows}
\end{split}
\end{equation}
\end{proof}

\begin{definition}{Least-Squares Line}{}
Let $\hat{\beta_0}$ and $\hat{\beta_1}$ be as defined in \ref{Least-Squares}。 The line defined by the equation $y = \hat{\beta_0} + \hat{\beta_1}x$ is called the least-squares line.

\end{definition}



\end{document}