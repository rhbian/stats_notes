%\documentclass[../special_distributions.tex]{subfiles}
\documentclass[../main.tex]{subfiles}
\begin{document}
\section{Gamma分布}

\begin{definition}{The Gamma Function}{}
For each positive number α, let the value $\Gamma(\alpha)$ be defined by
the following integral:
\begin{equation}\label{}
\Gamma(\alpha) = \int\limits_{0}^{\infty}x^{\alpha-1}e^{-x}dx
\end{equation}
The function $\Gamma$ defined by Eq. (5.7.2) for $\alpha > 0$ is called the gamma function.
\end{definition}

\begin{theorem}{}{}
if $\alpha > 1$, then
\begin{equation}\label{}
\Gamma(\alpha)=(\alpha-1)\Gamma(\alpha-1)
\end{equation}
\end{theorem}

\begin{proof}
We shall apply the method of integration by parts to the integral in Eq. (5.7.2).
If we let $u=x^{\alpha-1}$ and $dv=e^{-x}dx$, then $du=(\alpha-1)x^{\alpha-2}dx$ and $v=-e^{-x}$. Therefore,
\begin{equation}\label{}
\begin{split}
\Gamma(\alpha)&=\int_{0}^{\infty}udv=[uv]_0^{\infty}-\int_{0}^{\infty}vdu\\
&=[-x^{\alpha-1}e^{-x}]_0^{\infty} + (\alpha-1)\int_{0}^{\infty}x^{\alpha-2}e^{-x}dx\\
&=0+(\alpha-1)\Gamma(\alpha-1)
\end{split}
\end{equation}
\end{proof}


\begin{theorem}{}{}
For every positive integer n,
\begin{equation}\label{}
\Gamma(n)=(n-1)!
\end{equation}

\begin{proof}
\begin{equation}\label{}
\begin{split}
\Gamma(n)&=(n-1)\Gamma(n-1)\\
&=(n-1)(n-2)\Gamma(n-2)\\
&=(n-1)(n-2)\cdots(1)\Gamma(1)\\
&=(n-1)!
\end{split}
\end{equation}
\end{proof}
\end{theorem}


\begin{theorem}{}{}
For each α > 0 and each β > 0,
\begin{equation}\label{}
\int_{0}^{\infty}x^{\alpha-1}e^{-\beta x}dx=\frac{\Gamma(\alpha)}{\beta^\alpha}
\end{equation}
\end{theorem}
\begin{proof}
令$y=\beta x$,则有$x=y/\beta$,以及$dx = dy/\beta$。
\begin{equation}\label{}
\begin{split}
\begin{WithArrows}
\int_{0}^{\infty}x^{\alpha-1}e^{-\beta x}dx&=\int_{0}^{\infty} \frac{y^{\alpha-1}}{\beta^{\alpha-1}}e^{-y}\frac{1}{\beta}dy\\
&=\frac{1}{\beta^{\alpha}}\int_{0}^{\infty}y^{\alpha-1}e^{-y}dy\\
&=\frac{\Gamma(\alpha)}{\beta^{\alpha}}
\end{WithArrows}
\end{split}
\end{equation}
\end{proof}

\begin{definition}{Gamma Distributions}{}
Let $\alpha$ and $\beta$ be positive numbers. A random variable $X$ has the gamma distribution with parameters $\alpha$ and $\beta$ if $X$ has a continuous distribution for which the $p.d.f.$ is
%\begin{equation}\label{}
%f(x|\alpha, \beta)=\begin{cases}
%& \frac{\beta^\alpha}{\Gamma(\alpha)}x^{\alpha-1}e^{-\beta x} &b \quad\text{for $x > 0$}\\
%& 0 & b \quad\text{for $x \leqslant 0$}
%\end{cases}
%\end{equation}

\begin{equation}\label{}
f(x|\alpha, \beta)=\left\lbrace \begin{aligned}
& \frac{\beta^\alpha}{\Gamma(\alpha)}x^{\alpha-1}e^{-\beta x} &\quad\text{for $x > 0$}\\
& 0 &\quad\text{for $x \leqslant 0$}
\end{aligned}
\right.
\end{equation}
\end{definition}

\begin{theorem}{Moments}{}
Let $X$ have the gamma distribution with parameters $\alpha$ and $\beta$. For $k =
1, 2, \dots$
\begin{equation}\label{}
E(X^k)=\frac{\Gamma(\alpha+k)}{\beta^k\Gamma(\alpha)}=\frac{\alpha(\alpha+1)\cdots(\alpha+k-1)}{\beta^k}
\end{equation}

特别的,$E(X)=\frac{\alpha}{\beta}$,$Var(X)=\frac{\alpha}{\beta}$
\end{theorem}


\begin{proof}
\begin{equation}\label{}
\begin{split}
E(X^k)&= \int_{0}^{\infty} x^kf(x|\alpha, \beta)dx= \int_{0}^{\infty} x^k \cdot \frac{\beta^\alpha}{\Gamma(\alpha)}x^{\alpha-1} e^{-\beta x}\\
&=\frac{\beta^\alpha}{\Gamma(\alpha)} \int_{0}^{\infty}x^{\alpha+k-1}e^{-\beta x}dx\\
&=\frac{\beta^\alpha}{\Gamma(\alpha)} \cdot \frac{\Gamma(\alpha+k)}{\beta^{\alpha+k}}\\
&=\frac{\Gamma(\alpha+k)}{\Gamma(\alpha)\beta^k}
\end{split}
\end{equation}
因此,$E(X)=\frac{\Gamma(\alpha+1)}{\Gamma(\alpha)\beta}=\frac{\alpha}{\beta}$,$Var(X)=E(X^2)-E^2(X)=\frac{(\alpha+1)\alpha}{\beta^2}-\frac{\alpha^2}{\beta^2}=\frac{\alpha}{\beta^2}$
\end{proof}

% 矩生成函数
\begin{theorem}{Moment Generating Function}{}
Let X have the gamma distribution with parameters α
and β. The m.g.f. of X is
\begin{equation}\label{}
\psi(t)=(\frac{\beta}{\beta-t})^\alpha \quad\text{for $t<\beta$}
\end{equation}
\end{theorem}
\begin{proof}
\begin{equation}\label{}
\begin{split}
\psi(t)&=E(e^{tX})=\int_{0}^{\infty}e^{tx} \frac{\beta^\alpha}{\Gamma(\alpha)}x^{\alpha-1}e^{-\beta x}dx\\
&=\frac{\beta^\alpha}{\Gamma(\alpha)} \int_{0}^{\infty}x^{\alpha-1}e^{-(\beta-t)x}dx\\
&=\frac{\beta^\alpha}{\Gamma(\alpha)} \frac{(\Gamma(\alpha))}{(\beta-t)^\alpha}\\
&=(\frac{\beta}{(\beta-t)})^\alpha
\end{split}
\end{equation}
\end{proof}

% 可加性
\begin{theorem}{}{}
If the random variables X1, . . . , Xk are independent, and if Xi has the gamma
distribution with parameters αi and β (i = 1, . . . , k), then the sum X1 + . . . + Xk
has the gamma distribution with parameters α1 + . . . + αk and β.
\end{theorem}

\begin{proof}
If $\psi_i(t)$ denotes the m.g.f. of Xi , then it follows from Eq. (5.7.15) that for
i = 1, . . . , k,
\begin{equation}\label{}
\psi(t)=\prod\limits_{i=1}^{k}\psi_i(t)=(\frac{\beta}{\beta-t})^{\alpha_1+\cdots+\alpha_k} \quad\text{for $t < \beta$}
\end{equation}
The m.g.f. ψ can now be recognized as the m.g.f. of the gamma distribution with
parameters α1 + . . . + αk and β. Hence, the sum X1 + . . . + Xk must have this gamma
distribution.
\end{proof}
\end{document}