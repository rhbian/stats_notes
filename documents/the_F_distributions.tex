\documentclass[../main.tex]{subfiles}

\begin{document}

\begin{definition}{The F distributions}{}
The $F$ distributions. Let $Y$ and $W$ be independent random variables such that $Y$ has the $\chi^2$ distribution with m degrees of freedom and $W$ has the $\chi^2$ distribution with $n$ degrees of freedom, where $m$ and $n$ are given positive integers. Define a new random variable $X$ as follows:
\begin{equation}\label{}
X=\dfrac{Y/m}{W/n}
\end{equation}
Then the distribution of $X$ is called the $F$ distribution with $m$ and $n$ degrees of freedom.
\end{definition}

矩阵测试
\begin{equation}\label{}
\begin{bmatrix}
a_{11} & a_{12} & \dots & a_{1n}\\
a_{21} & a_{22} & \dots & a_{2n}\\
\vdots & \vdots   & \ddots & \vdots \\
a_{n1} & a_{n2} & \dots & a_{nn}
\end{bmatrix}
\end{equation}

设 $A$ 是 $n \times n$ 矩阵,$B$ 是 $n \times n$ 矩阵,则有:
\begin{equation*}
\det(AB) = \det(A) \det(B)^2
\end{equation*}

证明如下:

设 $C = AB$,$D_A = \det(A)$,$D_B = \det(B)$,$D_C = \det(C)$。

根据行列式的定义,$D_A$ 可以表示成 $n$ 个 $A$ 的行列式的乘积:

\begin{equation*}
D_A = \sum_{\sigma \in S_n} \operatorname{sgn}(\sigma) \prod_{i=1}^n A_{i,\sigma(i)}
\end{equation*}

类似地,$D_B$ 和 $D_C$ 也可以表示成对应矩阵元素的乘积之和。我们考虑 $D_C$ 的表达式,由于 $C = AB$,因此

\begin{equation*}
D_C = \sum_{\sigma \in S_n} \operatorname{sgn}(\sigma) \prod_{i=1}^n (AB)_{i,\sigma(i)}
\end{equation*}

根据矩阵乘法的定义,$(AB){i,j} = \sum{k=1}^n A_{i,k}B_{k,j}$。因此,我们可以将 $D_C$ 表示为:

\begin{align*}
D_C &= \sum_{\sigma \in S_n} \operatorname{sgn}(\sigma) \prod_{i=1}^n \sum_{k=1}^n A_{i,k}B_{k,\sigma(i)} \
&= \sum_{\sigma \in S_n} \operatorname{sgn}(\sigma) \sum_{k_1=1}^n \cdots \sum_{k_n=1}^n A_{1,k_1} \cdots A_{n,k_n} B_{k_1,\sigma(1)} \cdots B_{k_n,\sigma(n)}
\end{align*}

注意到上式中,$\sigma$ 是对称群 $S_n$ 中的一个置换,$k_1,\ldots,k_n$ 则是 $n$ 个独立的下标,因此对每个置换 $\sigma$,上式中总共有 $n!$ 个不同的 $k_1,\ldots,k_n$ 组合。将这些组合分为两类,一类是 $\sigma$ 是偶置换,另一类是 $\sigma$ 是奇置换。对于任意一个固定的组合 $k_1,\ldots,k_n$,置换 $\sigma$ 的奇偶性只有两种可能性,因此:

\begin{align*}
D_C &= \sum_{k_1=1}^n \cdots \sum_{k_n=1}^n A_{1,k_1} \cdots A_{n,k_n} \sum_{\sigma \in S_n} \operatorname{sgn}(\sigma) B_{k_1,\sigma(1)} \cdots B_{k_n,\sigma(n)} \
&= \sum_{k_1=1}^n \cdots \sum_{k_n=1}^n A_{1,k_1} \cdots A_{n,k_n} D_B
\end{align*}

因此,

\begin{align*}
\frac{D_C}{D_B} &= \sum_{k_1=1}^n \cdots \sum_{k_n=1}
\frac{D_C}{D_B} &= \sum_{k_1=1}^n \cdots \sum_{k_n=1}^n A_{1,k_1} \cdots A_{n,k_n} \
&= D_A
\end{align*}

也就是说,$\det(AB) = D_C = D_B \cdot D_A = \det(A) \cdot \det(B)^2$。证毕。
\end{document}