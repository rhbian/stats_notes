\documentclass[../main.tex]{subfiles}
\begin{document}
% 大数定律
\section{大数定律}
\begin{theorem}{Markov Inequality}{}
Suppose that $X$ is a random variable such that $p(X \geqslant 0) = 1$. Then
for every real number $t > 0$,
\begin{equation}\label{}
p(X \geqslant t)\leqslant \frac{E(X)}{t}
\end{equation}
\end{theorem}
\begin{proof}
离散情况下
\begin{equation}\label{}
E(X)=\sum_{x}xf(x)=\sum_{x<t}xf(x)+\sum_{x\geqslant t}xf(x)
\end{equation}
由于$X\geqslant 0$,所有项都大于0。因此
\begin{equation}\label{}
E(X)\geqslant\sum_{x\geqslant t}xf(x)\geqslant \sum_{x\geqslant t}tf(x)=t\cdot p(X\geqslant t)
\end{equation}

连续情况下
\begin{equation}\label{}
\begin{split}
E(X)&=\int_{0}^{\infty}xf(x)dx\\
&=\int_{0}^{t}xf(x)dx + \int_{t}^{\infty}xf(x)dx\\
&\geqslant \int_{t}^{\infty}xf(x)dx\\
&\geqslant \int_{t}^{\infty}tf(x)dx\\
&=t\cdot p(X\geqslant t)
\end{split}
\end{equation}
\end{proof}

\begin{theorem}{Chebyshev Inequality}{}
Let $X$ be a random variable for which $Var(X)$ exists. Then for every number $t > 0$,
\begin{equation}\label{}
p(|X-E(X)|\geqslant t)\leqslant \frac{Var(X)}{t^2}
\end{equation}
\end{theorem}

\begin{proof}
令$Y=[X-E(X)^2]$,则$E(Y)=Var(X)$,
\begin{equation}\label{}
p(|X-E(X)|\geqslant t)=p(Y\geqslant t^2)\leqslant \frac{E(Y)}{t^2}=\frac{Var(X)}{t^2}
\end{equation}
\end{proof}

\begin{theorem}{Mean and Variance of the Sample Mean}{}
Let $X_1, \dots , X_n$ be a random sample from a distribution with mean $\mu$ and variance $\sigma^2$ . Let $\bar{X}_n$ be the sample mean. Then
$E(\bar{X}_n)=\mu$ and $Var(\bar{X}_n)=\frac{\sigma^2}{n}$.
\end{theorem}

\begin{definition}{Convergence in Probability}{}
A sequence $Z_1, Z_2, \dots$ of random variables converges to $b$ in probability if for every number $\varepsilon > 0$,
\begin{equation}\label{}
\lim\limits_{n\rightarrow \infty}p(|Z_n-b|<\varepsilon)=1
\end{equation}

This property is denoted by
\begin{equation}\label{}
Z_n \xrightarrow{p} b
\end{equation}
and is sometimes stated simply as $Z_n$ converges to b in probability.
\end{definition}

\begin{theorem}{Law of Large Numbers}{}
Suppose that $X_1,\dots, X_n$ form a random sample from a distribution for which the mean is $\mu$ and for which the variance is finite. Let $\bar{X}_n$ denote the sample mean. Then
\begin{equation}\label{}
\bar{X}_n \xrightarrow{\enskip p \enskip} \mu
\end{equation}
\end{theorem}

\begin{proof}
Let the variance of each $X_i$ be $\sigma^2$. It then follows from the Chebyshev inequality that for every number $\varepsilon > 0$,
\begin{equation}\label{}
p(|\bar{X}_n-\mu|<\varepsilon)\geqslant 1-\frac{\sigma^2}{n\varepsilon^2}
\end{equation}
因此,
\begin{equation}\label{}
\lim_{n\rightarrow \infty} p(|\bar{X}_n - \mu|<\varepsilon)=1
\end{equation}
which means that $\bar{X}_n \xrightarrow{\enskip p \enskip} \mu$.
\end{proof}

\begin{theorem}{Continuous Functions of Random Variables}{}
If $Z_n \xrightarrow{\enskip p\enskip} b$, and if $g(z)$ is a function that is continuous at $z = b$, then$ g(Z_n) \xrightarrow{\enskip p\enskip} g(b)$.
\end{theorem}
\end{document}